% Das ist mein Template für die Präsentationen an der DHBW Stuttgart. Nicht perfekt, also falls ihr Verbesserungsvorschläge habt, stellt gerne einen Pull-Request. https://github.com/NikomitK/TX000_Template
% Bitte lasst auch einen star auf github da, danke.
% Damit es richtig funktioniert, müssen die Schriftarten Fira Sans und Fira Mono installiert sein.
\documentclass[ngerman,14pt,aspectratio=1610]{beamer}

% Imports
\usepackage[utf8]{inputenc}
\usepackage[T1]{fontenc}
\usepackage{babel}
\usepackage{graphicx}
\usepackage{multicol}
\usepackage{textpos} %logo in frametitle
\usepackage{ifthen}
\usepackage{totcount}

% Startpfad für Bilder setzen
\graphicspath{ {./images/} }

% DHBW Stuttgart logo mit transparentem Hintergrund
\newcommand{\dhbwlogo}{\includegraphics[width=5cm]{dhbw_lang_trans}}

% Subsections zählen
\def\secname{Gliederung}
\newcommand{\mysection}[1]{
	\section{#1}
	\newtotcounter{#1}
	\def\secname{#1}
}

\newcommand{\mysubsection}[1]{
	\subsection{#1} 
	\stepcounter{\secname}
}

% Nummern bei Frametitle
\newcommand{\secpagnum}{\insertsectionnumber.\thesubsection}
\newcommand{\lastpagenum}{1}
\let\oldframetitle\frametitle
\renewcommand{\frametitle}[1]{
	\ifthenelse{\equal{\lastpagenum}{\insertslidenumber}}{
		\mysubsection{}
		\renewcommand{\lastpagenum}{\insertslidenumber}
	}{
		% Hier soll nichts passieren, damit bei gleichen slides der title nicht neu gesetzt wird
	}
	
	\ifthenelse{\totvalue{\secname}>1}{
	\oldframetitle{\secpagnum~#1}
	}{
	\oldframetitle{\insertsectionnumber~#1}
	}
}

% Multicol column balancing ausschalten
\raggedcolumns

% Basis-Theme
\usetheme[progressbar=frametitle, block=fill, sectionpage=none]{metropolis} % Hier kann die progressbar deaktiviert/replaziert werden
\setbeamertemplate{frame numbering}[counter]
\useoutertheme{metropolis}
\useinnertheme{metropolis}
\usefonttheme{metropolis}
\setbeamercolor{background canvas}{bg=white}

% Sections im ToC Nummerieren
\setbeamertemplate{section in toc}[sections numbered]
\setbeamertemplate{subsection in toc}[subsections numbered]

% DHBW Farben
\definecolor{dhbw_grau}{RGB}{93,104,110}
\definecolor{dhbw_rot}{RGB}{227,6,19}
\definecolor{dunkelgrau}{RGB}{41,55,67}
\definecolor{hellgrau}{RGB}{239,241,242}

% DHBW Farben einbauen
\setbeamercolor{progress bar}{fg=dhbw_rot, bg=hellgrau}
\setbeamercolor{normal text}{fg=dhbw_grau} %bg hier macht block bg
\setbeamercolor{alerted text}{fg=dhbw_rot}
\setbeamercolor{frametitle}{fg=dhbw_rot,bg=hellgrau}
\setbeamercolor{title}{fg=dhbw_rot}
\setbeamercolor{subtitle}{fg=dunkelgrau}
\setbeamercolor{section title}{fg=dhbw_rot}
\setbeamercolor{institute}{fg=dhbw_rot}

% Logo in frametitle
\addtobeamertemplate{frametitle}{}{%
	\begin{textblock*}{5cm}(\textwidth-4cm,-1.3cm)
		\dhbwlogo
\end{textblock*}}

% Datum in der Fußzeile
\setbeamertemplate{frame footer}{
	\hskip 2.5em \insertdate
}

% Daten für die Titelseite
\title{Präsentationstitel}
%\subtitle{subtitle}
\author{Vorname Nachname}
\institute[DHBW Stuttgart]{Duale Hochschule Baden-Württemberg Stuttgart}
\date{24.09.2024}	
\titlegraphic { 
	\begin{tikzpicture}[overlay,remember picture]
		\node[right=0.5cm] at (current page.155){
			\dhbwlogo
		};
	\end{tikzpicture}
}

\regtotcounter{section}
\regtotcounter{subsection}

\begin{document}
	
	\begin{frame}[plain,noframenumbering]
		\titlepage
	\end{frame}

	\mysection{Gliederung}
	
	\begin{frame}[t]{Gliederung} \vspace{20pt}
		\begin{columns}[T, onlytextwidth]
			\column{0.5\textwidth}
			\tableofcontents
			\column{0.5\textwidth}
		\end{columns}
	\end{frame}
	
	\metroset{sectionpage=simple} % Erst hier definiert, damit die Gliederung keine Sectionpage hat
	\mysection{Mach erschmal die basics}
	
	\begin{frame}[t]{``Animationen''} \vspace{20pt}
		\begin{itemize}
			\item <1-> Hallo Welt
			\item <2-> Beispielstichpunkt ab Schritt 2
			\item <-2> Beispielschritt bis inkl. Schritt 2
		\end{itemize}
		
		\onslide<2,4>{onslide<x> ist zwar unsichtbar bei $Seite \neq x$} \onslide<3->{Der Platz wird aber trotzdem blockiert}
		
		\only<2>{only<x> dagegen verbraucht den Platz}\only<3>{nur, wenn der Text tatsächlich sichtbar ist}
		
		Hier geht auch \alert<1-3>{ALARRRMMMM}, \textbf<2-3>{Fett},\textit<3>{Italienisch}, \alert<4>{\textit<4>{\textbf<4>{etc...}}}
	\end{frame}

	
	\metroset{sectionpage=simple}
	\mysection{Spalten}

	\begin{frame}[t]{Zwei Spalten} \vspace{20pt}
		\begin{columns}[T, onlytextwidth]
			\column{0.45\textwidth}
				\begin{itemize}
					\item test1
					\item test2
					\item test3
				\end{itemize}
				\begin{block}{Wichtig!}
					Bei den Columns muss das große T als Argument verwendet werden
				\end{block}
			
			\column{0.45\textwidth}
				\includegraphics[width=\linewidth]{testbild}
		\end{columns}
	\end{frame}

		
	\begin{frame}[t]{Drei Spalten} \vspace{20pt}
		\begin{columns}[T, onlytextwidth]
			\column{0.33\textwidth}
			Spalte 1
			\column{0.33\textwidth}
			Spalte 2
			\column{0.33\textwidth}
			Spalte 3
		\end{columns}
	\end{frame}
	
	\begin{frame}[t]{Automatisch verteilte Spalten} \vspace{20pt}
		\begin{multicols*}{3}
			[
			\textbf{Multicols} \\
			Das Package $multicols$ ermöglicht es, anzugeben, auf wie viele Spalten ein bestimmter Inhalt automatisch verteilt werden soll.
			]
			\begin{enumerate}[a)]
				\item test1
				\item test2
				\item test3
				\item test1
				\item test2
				\item test3
				\item test1
				\item test2
				\item test3
				\item test1
				\item test2
			\end{enumerate}
		\end{multicols*}
	\end{frame}


\end{document}