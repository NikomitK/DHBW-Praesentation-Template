\documentclass[ngerman,14pt,aspectratio=1610]{beamer}

% Imports
\usepackage[utf8]{inputenc}
\usepackage[T1]{fontenc}
\usepackage{babel}
\usepackage{graphicx}
\usepackage{multicol}
\usepackage{textpos} %logo in frametitle

% Startpfad für Bilder setzen
\graphicspath{ {./images/} }

% DHBW Stuttgart logo mit transparentem Hintergrund
\newcommand{\dhbwlogo}{\includegraphics[width=5cm]{dhbw_lang_trans}}

% Multicol column balancing ausschalten
\raggedcolumns

% Basis-Theme
\usetheme[progressbar=frametitle]{metropolis} % Hier kann die progressbar deaktiviert/replaziert werden
\setbeamertemplate{frame numbering}[counter]
\useoutertheme{metropolis}
\useinnertheme{metropolis}
\usefonttheme{metropolis}
\setbeamercolor{background canvas}{bg=white}

% DHBW Farben
\definecolor{dhbw_grau}{RGB}{93,104,110}
\definecolor{dhbw_rot}{RGB}{227,6,19}
\definecolor{dunkelgrau}{RGB}{41,55,67}
\definecolor{hellgrau}{RGB}{239,241,242}

% DHBW Farben einbauen
\setbeamercolor{progress bar}{fg=dhbw_rot, bg=hellgrau}
\setbeamercolor{normal text}{fg=dhbw_grau} %bg hier macht block bg
\setbeamercolor{alerted text}{fg=dhbw_rot}
\setbeamercolor{frametitle}{fg=dhbw_rot,bg=hellgrau}
\setbeamercolor{title}{fg=dhbw_rot}
\setbeamercolor{subtitle}{fg=dunkelgrau}
\setbeamercolor{section title}{fg=dhbw_rot}
\setbeamercolor{institute}{fg=dhbw_rot}

% Logo in frametitle
\addtobeamertemplate{frametitle}{}{%
	\begin{textblock*}{5cm}(\textwidth-4cm,-1.3cm)
		\dhbwlogo
\end{textblock*}}

% Datum in der Fußzeile
\setbeamertemplate{frame footer}{
	\hskip 2.5em \insertdate
}

% Daten für die Titelseite
\title{Implementierung Erweitertes Audit Logging}
%\subtitle{subtitle}
\author{Nikolas Bodenmüller}
\institute[DHBW Stuttgart]{Duale Hochschule Baden-Württemberg Stuttgart}
\date{24.09.2024}

\begin{document}
	\metroset{block=fill, sectionpage=simple}
	
	\titlegraphic { 
		\begin{tikzpicture}[overlay,remember picture]
			\node[right=0.5cm] at (current page.155){
				\dhbwlogo
			};
		\end{tikzpicture}
	}
	
	\begin{frame}
		\titlepage
	\end{frame}
	
	
		\begin{itemize}
			\item <1-> Hallo Welt
			\item <2-> Beispielstichpunkt ab Schritt 2
			\item <-2> Beispielschritt bis inkl. Schritt 2
		\end{itemize}
		
		\onslide<2,4>{onslide<x> ist zwar unsichtbar bei $Seite \neq x$} \onslide<3->{Der Platz wird aber trotzdem blockiert}
		
		\only<2>{only<x> dagegen verbraucht den Platz}\only<3>{nur, wenn der Text tatsächlich sichtbar ist}
		
		Hier geht auch \alert<1-3>{ALARRRMMMM}, \textbf<2-3>{Fett},\textit<3>{Italienisch}, \alert<4>{\textit<4>{\textbf<4>{etc...}}}
	\end{frame}

	

	
	\begin{frame}[t]{Zwei Spalten} \vspace{10pt}
		\begin{columns}[T, onlytextwidth]
			\begin{column}{0.45\textwidth}
				\begin{itemize}
					\item test1
					\item test2
					\item test3
				\end{itemize}
				\begin{block}{Wichtig!}
				Bei den Columns muss das große T als Argument verwendet werden
				\end{block}
			\end{column}
			
			\begin{column}{0.45\textwidth}
				\includegraphics[width=\linewidth]{testbild}
			\end{column}
		\end{columns}
	\end{frame}

		
	\begin{frame}[t]{Drei Spalten} \vspace{10pt}
		\begin{columns}[onlytextwidth]
			\column{0.33\textwidth}
			Spalte 1
			\column{0.33\textwidth}
			Spalte 2
			\column{0.33\textwidth}
			Spalte 3
		\end{columns}
	\end{frame}
	
	\begin{frame}{Automatisch verteilte Spalten}
		\begin{multicols*}{3}
			[
			\textbf{Multicols} \\
			Das Package $multicols$ ermöglicht es, anzugeben, auf wie viele Spalten ein bestimmter Inhalt automatisch verteilt werden soll.
			]
			\begin{enumerate}[a)]
				\item test1
				\item test2
				\item test3
				\item test1
				\item test2
				\item test3
				\item test1
				\item test2
				\item test3
				\item test1
				\item test2
			\end{enumerate}
		\end{multicols*}
	\end{frame}
	

\end{document}